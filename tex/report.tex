\documentclass[11pt,a4paper]{article}
\usepackage{epsfig}
\usepackage{amssymb}
\usepackage{latexsym}
%\usepackage{epic}
%\usepackage{eepic}
\usepackage{amsmath}
\usepackage{verbatim}

\textwidth 146 mm
\textheight 230 mm
\oddsidemargin 7mm \evensidemargin -1mm \topmargin -4mm

\author{Jonatan Kallus (kallus@student.chalmers.se)\\
Joel Wilsson (wjoel@student.chalmers.se)}
\date{\today}

\title{Modelling traffic flow}

\begin{document}
\maketitle

\section{Introduction}
Traffic flow can be modelled at a macroscopic level, where traffic flow is
described as a fluid, or at the microscropic level, where the behaviour of
individual cars is described. When modelling traffic as a fluid, there has
to be quite a lot of cars and the gaps in between them must be small. Only
then is it possible to get a reasonable approximation by viewing the traffic
as a continuum. \cite{aziz}

In this project we use a microscopic model, and by stochastic simulations
we gather data about the global characteristics of the system. 

\section{Model}
\subsection{Generating a random city}
\subsection{Parameters}

\subsection{Car types}
\subsubsection{Random cars}
\subsubsection{Cars with goals}
\subsubsubsection{Path finding}
In order to get a path from the car's current path to the destination we use the
A* path finding algorithm \cite{astar}, which is implemented in the GRaph Theory
for Ruby library (GRATR) \cite{gratr}. In order to be able to use GRATR, a
directed graph is constructed from the map. In this graph, there is a node for
each path in the map and an edge from one node to another if the path
represented by the first node leads to the path represented by the second node.

The A* algorithm does not always find the shortest path, but manual inspection
of the paths chosen for cars showed that it never picked paths that were
unreasonably long. Since we don't require the shortest path, this is acceptable.
Indeed, one could imagine that a driver has some personal reasons for choosing
a slightly longer path, perhaps because he or she enjoys the scenery of this
path.

\section{ODE description of car velocity}
\subsection{Gap acceptance}
\subsection{Stability}
\subsection{Phase portrait}

\section{Fundamental characteristics of road traffic}
\subsection{The fundamental diagram of road traffic}

\section{Modelling traffic flow as a fluid}

\section{How parameters affect throughput}

\begin{thebibliography}{99}
\bibitem{haight} Frank A. Haight
{\it Mathematical theories of traffic flow} {\bf 1963}

\bibitem{aziz} Denos C. Aziz
{\it Traffic Theory} {\bf 2002}

\bibitem{gratr} Shawn Patrick Garbett
{\it GRAph Theory in Ruby (GRATR)} {\bf 2006}

\bibitem{astar} P. E. Hart, N. J. Nilsson, B. Raphael
{\it A Formal Basis for the Heuristic Determination of Minimum Cost Paths} {\bf 1968}
%\bibitem{wiki} Wikipedia
%{\it Entry for the Ising model} {\bf 2011}

\end{thebibliography}

\end{document}
