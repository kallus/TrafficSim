\documentclass[11pt,a4paper]{report}
\usepackage{epsfig}
\usepackage{amssymb}
\usepackage{latexsym}
%\usepackage{epic}
%\usepackage{eepic}
\usepackage{amsmath}
\usepackage{verbatim}

% Stolen settings (unknown origin):
% Alter some LaTeX defaults for better treatment of figures:
% See p.105 of "TeX Unbound" for suggested values.
% See pp. 199-200 of Lamport's "LaTeX" book for details.
%   General parameters, for ALL pages:
\renewcommand{\topfraction}{0.9}	% max fraction of floats at top
\renewcommand{\bottomfraction}{0.8}	% max fraction of floats at bottom
%   Parameters for TEXT pages (not float pages):
\setcounter{topnumber}{2}
\setcounter{bottomnumber}{2}
\setcounter{totalnumber}{4}     % 2 may work better
\setcounter{dbltopnumber}{2}    % for 2-column pages
\renewcommand{\dbltopfraction}{0.9}	% fit big float above 2-col. text
\renewcommand{\textfraction}{0.07}	% allow minimal text w. figs
%   Parameters for FLOAT pages (not text pages):
\renewcommand{\floatpagefraction}{0.7}	% require fuller float pages
% N.B.: floatpagefraction MUST be less than topfraction !!
\renewcommand{\dblfloatpagefraction}{0.7}	% require fuller float pages

% remember to use [htp] or [htpb] for placement

\author{Jonatan Kallus (kallus@student.chalmers.se)\\
Joel Wilsson (wjoel@student.chalmers.se)}
\date{\today}

\title{Modelling traffic flow}

\begin{document}
\maketitle

\section{Introduction}
Traffic flow can be modelled at a macroscopic level, where traffic flow is
described as a fluid, or at the microscropic level, where the behaviour of
individual cars is described. When modelling traffic as a fluid, there has
to be quite a lot of cars and the gaps in between them must be small. Only
then is it possible to get a reasonable approximation by viewing the traffic
as a continuum. [ref: aziz]

In this project we use a microscopic model, and by stochastic simulations
we gather data about the global characteristics of the system.

\section{Model}
\subsection{Generating a random city}
\subsection{Parameters}

\section{Car types}
\subsection{Random cars}
\subsection{Cars with goals}
\subsubsection{Path finding}

\section{ODE description of car velocity}
\subsection{Gap acceptance}
\subsection{Stability}
\subsection{Phase portrait}

\section{Modelling traffic flow as a fluid}

\section{How parameters affect throughput}

\begin{thebibliography}{99}
%\bibitem{lindgren} Kristian Lindgren
%{\it Information Theory for Complex Systems, Lecture Notes} {\bf 2008}

%\bibitem{wiki} Wikipedia
%{\it Entry for the Ising model} {\bf 2011}

\end{thebibliography}

\end{document}
