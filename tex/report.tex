\documentclass[11pt,a4paper]{article}
\usepackage{epsfig}
\usepackage{amssymb}
\usepackage{latexsym}
%\usepackage{epic}
%\usepackage{eepic}
\usepackage{amsmath}
\usepackage{verbatim}

\textwidth 146 mm
\textheight 230 mm
\oddsidemargin 7mm \evensidemargin -1mm \topmargin -4mm

\author{Jonatan Kallus (kallus@student.chalmers.se)\\
Joel Wilsson (wjoel@student.chalmers.se)}
\date{\today}

\title{Modelling traffic flow}

\begin{document}
\maketitle

\section{Introduction}
modelling traffic flow

dynamic agent based model

\section{Model}
The model consists of a \textit{city} and cars of two different types. In this model the city is a non-complete lattice of roads with intersections. All roads have two lanes, one in each direction. The two types of cars differ only in the way their direction is chosen. One type of cars have an assigned goal and uses a path finding algorithm to approach that goal. The other type of car choses a random direction each time it encounters an intersection.
\subsection{Generating a random city}
A random city is generated by taking a two dimensional lattice with a certain \textit{width} and \textit{height}. The lattice is fully connected initially, with a four-way intersection at each node in the lattice. As a second step connections are removed in the lattice until a certain fraction of the original edges remains. This fraction parameter will be called \textit{connectivity}. This removal of edges causes some of the four-way intersections to become three-way intersections, turns or ordinary straight roads.

The city is created in this way to be more realistic and to show more interesting characteristics possible causing some road parts to be used more often and others to be used less often.

In summary, the generation of the city for the model contributes with three parameters for the model: width, height and connectivity.
\subsection{Car types}
As mentioned earlier, the model has two types of cars, random cars and cars with goals. The two types are choosen to make the model as simple as possible, but to still show some interesting and measurable behavior.
\subsubsection{Random cars}
Random cars are used in the model to add noise and to make cars occur in all parts of the city. The number of random cars will be constant over time, random cars are not created after the initialization of the model and they do not leave the city. The number of random cars is an input parameter to the model.
\subsubsection{Cars with goals}

\subsection{Parameters}
In total, the input parameters for the model are XX. The parameters YY will be kept constant at values that produces a basis for interesting and measurable behaviors. ZZ will be varied and their effect on the model will be examined.

\section{ODE description of car velocity}
\subsection{Gap acceptance}
\subsection{Stability}
\subsection{Phase portrait}

\section{Fundamental characteristics of road traffic}
\subsection{The fundamental diagram of road traffic}

\section{Modelling traffic flow as a fluid}

\section{How parameters affect throughput}

\begin{thebibliography}{99}
\bibitem{haight} Frank A. Haight
{\it Mathematical theories of traffic flow} {\bf 1963}

%\bibitem{wiki} Wikipedia
%{\it Entry for the Ising model} {\bf 2011}

\end{thebibliography}

\end{document}
